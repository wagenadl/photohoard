\documentclass[11pt]{report}
\usepackage[utf8]{inputenc}
\usepackage{mathptm}
\usepackage{newcent}
\usepackage[letterpaper]{geometry}
\usepackage[perpage]{footmisc}
\usepackage{wasysym}
\usepackage{amssymb}
\usepackage{footnote}
\makesavenoteenv{tabular}
\begin{document}
\thispagestyle{empty}
\begin{centering}
  {\Huge Photohoard}
  \vskip30pt

  {\Large Keeping your digital photos together}
  \vskip60pt

  {\large By Daniel A. Wagenaar}
  \vfill
  
  {Copyright (c) 2013--2018}
  
\end{centering}
\pagebreak
~
\vfill
\noindent Copyright (C) 2013--2018 Daniel A. Wagenaar\medskip

``Photohoard'' is free software: you can redistribute it and/or modify
it under the terms of the GNU General Public License as published by
the Free Software Foundation, either version 3 of the License, or
(at your option) any later version.

This program is distributed in the hope that it will be useful,
but WITHOUT ANY WARRANTY; without even the implied warranty of
MERCHANTABILITY or FITNESS FOR A PARTICULAR PURPOSE.  See the
GNU General Public License for more details.

You should have received a copy of the GNU General Public License
along with this program.  If not, see http://www.gnu.org/licenses.
\pagebreak

\chapter{Introduction}

If you are like me, you take\footnote{And, if you are lucky
  \emph{make} a few as well.} several thousand photos every year. You
don't have the time (or inclination) to organize them in a particular
way, so they end up in the digital equivalent of a shoebox. Then, your
child's school needs a few family pictures for some project and you
end up desparate searching through the folders. Very unpleasant,
because even the thumbnails take time to load when you have thousands
of them to look at.

\section{Organization}

Photohoard is primarily intended to ease this pain. It does not force
you to spend any time organizing or tagging your photos (though you
can), but simply presents them in a timeline. It also caches
previews at various scales, so it is lightningly fast at scrolling
through thousands of images.

Photohoard allows you to apply ``color labels'', ``status flags'',
``star ratings,'' and
arbitrary ``tags'' to your images and even to organize them into
``collections.''  To be honest, I use the facilities
minimally. Just having all my photos in a timeline is the most
important thing.

\section{Nondestructive editing}

In addition to organizing, Photohoard helps you develop your
photos. It contains a fully nondestructive image editing pipeline with
tools like:
\begin{itemize}
\item Cropping and rotating
\item Perspective correction
\item Exposure and curve correction
\item White balance correction
\item Unsharp masking and local contrast enhancement
\end{itemize}
All of the curve and color tools operate in appropriate color spaces,
mostly IPT.\footnote{Ebner, Fritz, 1998. \emph{Derivation and modelling hue uniformity and
development of the IPT color space.} Thesis. Rochester Institute
of Technology. Accessed from ResearchGate.net.} Moreover, Photohoard is fully color managed.

At present, Photohoard's tools are applied to the entire photo:
Regional masks are not yet supported, though they are being worked on
for a future release.

\section{Alternatives}

Of course, there are several alternatives to Photohoard. Just to
mention a few that work with Linux:
\begin{itemize}
  \item {\bf Shotwell} is a default application for organizing photos
    in Ubuntu and GNOME. At last check, its editing tools were limited.
  \item {\bf Digikam} has excellent organization facilities. It also
    has strong editing capabilities, but they are not nondestructive in the
    usual sense: Edited versions must be saved to a new file.
    \item {\bf Darktable} has the best editing capabilities, far more
      advanced than Photohoard. It is less suitable for maintaining
      large collections, because its thumbnailing system does not
      cache arbitrarily large numbers of files.
      \item {\bf Google Photos} offers organization and many of the
        same editing tools as Photohoard, but keeps your photos in the
        corporate cloud.
  \end{itemize}

\chapter{Installation}

If you are using Ubuntu, you can probably simply install a
``.deb'' package from the home page. To install from source, clone the
git archive from github (http://www.github.com/wagenadl/photohoard)
and install the requirements: qt5, lcms2, libexiv2, opencv-imgproc,
x11-xcb, and asciidoc (for the man page). Then, simply type ``make''
and ``sudo make install''.

\chapter{Organizing and viewing your photos}

\section{Establishing your library}
Photohoard keeps metadata about your pictures (including timestamps,
exposure information, and camera details) as well as all your edits in
a database file. It does not touch your original image
files.\footnote{Except during the ``Purge Rejects'' operation, which
  is documented below.} This database needs to be setup the first time
you use the program. We'll assume that you already have a set
of shoeboxes full of photos. Perhaps they are stored in various
subfolders of
/home/me/Pictures as well as in /data/more/Photos. The first time you
run Photohoard, you need to tell it about these locations by clicking
the ``Add new folder tree'' icon in the top left of the window (or
press Control+Shift+R). You can do this for as many locations as you
want. Subfolders are automatically included. And you only have to do
it once.

Adding a large tree to Photohoard takes a while, because Photohoard
reads every single image file to extract metadata and to create small
preview images. This operation completes in the background, so you
don't need to wait for it.

Note that Photohoard does not copy or move your original image
files. It simply creates references to them in its database. Just to
be clear: That means that you must not delete the original files. (If
you do delete them, Photohoard will drop them from its database next
time you rescan the tree; more on this below.)

\subsection{Adding more photos}

There are several ways to add more photos to Photohoard.

\subsubsection{From a camera}

If your photos are on a card from a digital camera, the easiest thing
to do is to insert that card in your computer, make sure it gets
mounted (which GNOME and Cinnamon do automatically), and click the
``Import from camera or card'' icon (or press Control+I). A dialog
window pops up in which you can choose where the photos will be
stored. (If Photohoard complains that ``No removable media can be
found,'' it usually suffices to open the card in a Files window.)

This same procedure also works for many cameras if you attach it to
your computer with a USB cable.

In the ``Import'' dialog, you can also specify that Photohoard should
delete the files from the card after importing, or move them to a
backup location. This location is a folder called
``photohoard-backup'' in the root of your card. (This is my preferred
method. I delete that backup just prior to the next time I import
photos, in the secure knowledge that I have backed up my laptop since
the time I previously imported photos.)

Photohoard cannot handle movie files, but if there are any movie files
on your card, it does offer to move them over to your computer. (I have
to admit I still keep movie files in a digital cardboard box.)

\subsubsection{From elsewhere}

If your image files are already on your computer, you can simply drag
them into Photohoard. You will be given the choice of adding their
location to the set of ``folder trees,'' or to move them into one of
the previously incorporated trees.

Alternatively, you can place the images inside a folder tree that
Photohoard already knows about. Photohoard will not automatically
discover that you did that, but you can tell it to check all of its folder
trees for new or modified files by clicking the ``Rescan folders''
icon (or pressing Control+R). This operation will take a while to
complete, but runs in the background so you don't have to wait for it.

Of course, you can also add a new folder tree to Photohoard at any
time.

\section{Browsing: Using the date strip}

In default mode, Photohoard displays a ``date strip'' of photos on the left,
one ``current'' photo in the center, and a series of adjustment
sliders on the right. The date strip is organized into ``epochs'' of
various lengths. An epoch can be: a decade, a year, a month, a day, an
hour, or ten minutes. To open an epoch, simply click on the tile  that
represents that epoch. If an epoch contains more than a few photos,
sub-epochs contained in it stay collapsed until you click on their
tiles. To uncollapse an epoch all at once, hold Shift while
clicking. To recollapse an epoch, simply click on the tab by the
epoch. (Try it; it is easier to navigate than it sounds.)

\section{Browsing: By folder}



\section{Marking}

Photohoard supports several methods to mark up your photos to aid
organization. These methods are:
\begin{description}
  \item[Status]Each photo can be marked either ``new,'' ``undecided,''
    ``accepted,'' or ``rejected.'' Photos are automatically marked
    ``new'' upon import, and ``undecided'' upon first view. Press
    Control+G to mark a photo as ``accepted,'' Control+X to mark a
    photo as ``rejected,'' or Control+U to mark a photo as
    ``undecided.'' Status is indicated by a small colored mark on the
    top right of the image tile.
  \item[Color label]Each photo can be marked with a color that is
    applied to the border of the image tile. Use Control+1 through
    Control+5 to apply a color label, Control+0 to remove it.
  \item[Star rating]Each photo can be marked with a rating of 1
    through 5 stars by pressing Alt+1 through Alt+5. Use Alt+0 to
    remove the rating.
    The star rating is indicated on the bottom left of the image tile.
  \item[Tags]Any number of arbitrary textual tags can be attached to photos.
\end{description}

\subsection{Tagging}

To apply a tag to an image or selection of images, simply type a tag
name in the ``New tag...'' box near the bottom right of the
window. While typing, the name of your tag will appear in deep red if it is
truly new, in green if it is a unique abbreviation of an existing tag,
or in black if it is not unique. Press Enter to apply the tag. This is
only possible if what you typed is either truly new or a unique match to an
existing tag. In the latter case, the tag name will be automatically expanded.

Tags can be organized in a hierarchy as well. For instance, you might
have a tag ``Places:Cities:Los Angeles.'' If such a tag exists, you
could refer to it using abbreviations like ``Los Ang,'' ``Cit:Los,''
``Pl:C:L,'' or ``Pla::Los,'' assuming you don't have a
``Places:Cities:Los Alamos'' tag.

The ``New tag...'' box is part of a larger ``Tags'' panel that
can be used to manage tags more broadly. Any tag that applies to all
images in the current selection are shown in black. Tags that apply to
only some images in the current selection are shown in blue. If you
hover over a tag, a small ``--'' icon appears that removes the tag
from all images in the selection. For blue tags, there will also be a
small ``+'' icon that applies the tag to all images in the selection.

The ``Tags'' panel also features a small ``...'' icon at the bottom
right. Click this to open the tag manager in which you can browse your
tag hierarchy, apply or remove tags from images, and create or delete
tags entirely. (Tags can only be deleted if they are not attached to
any images.)

\subsection{Collections}

Photohoard keeps all of your photos in a single database. But you may
have several very distinct bodies of photographic work. For instance, I have
family pictures, work-related photos, and images related to my wife's
art. Photohoard accommodates this situation by allowing you to group
images into ``collections.'' A collection is nothing but a regular
``tag'' inside the ``Collections'' tag group. When you first started
Photohoard, a collection named ``Family photos'' was automatically
created. This collection corresponds to a tag called
``Collections:Family photos.''  New collections are created with the
tag manager. There is nothing special about collections as opposed to
other tags, except that the ``Filter'' dialog has a special section
for filtering based on collection.

\section{Viewing a subset of photos}

Collapsing part of the date strip is one way to view only a subset
of photos, but it may be helpful to show only photos that share a
certain tag, where taken with a particular camera, or have a color label
attached. These criteria (and several more) can all be used to limit which
images are shown in the date strip. To do so, open the Filter dialog
by clicking the Filter icon or pressing Control+F. Several criteria
may be applied at once.

At present, it is not possible to
show photos \emph{not} made with a certain camera, or \emph{not}
tagged with a certain tag.

\chapter{Editing your photos}

\chapter{Exporting your photos}


\chapter{Technical information}
The first time you start Photohoard, it creates a database in
\$\{HOME\}/.local/photohoard/default.db. At present, it is not possible to change
that location, although you can move the file somewhere else and use a
link to help Photohoard find it. The database file doesn't get overly
big: 14 MB for my collection of 55,000 photos.

Photohoard also creates a cache for previews in
\$\{HOME\}/.cache/photohoard/default-cache. This does get fairly big:
6 GB for my collection. Nevertheless, it is helpful to have the cache
on a fast medium like an SSD.

Photohoard can, in principle, reconstruct the cache if it gets lost,
though it can take many hours. The database file is irreplaceable
and should be included in regular backups.

\end{document}
